\section{Future Work}
\label{s:future-work}

In our attempt to formulate a clear and concise vision for the \oscoin{}
protocol, we deliberately left certain questions unanswered. We will
briefly state these here and attempt to address them in the context of future work:

\begin{itemize}
\item As posed in Section \ref{s:osrank}, the question of information
reliability is crucial.  We are actively researching oracle-based solutions to
this problem which could provide economic incentives for users in the network
to flag suspicious projects, as well as further disincentive projects to cheat,
by allowing the protocol to slash their unvested (\S\ref{s:vesting}) reward
balance when dishonest behavior is detected.

\item In Section \ref{s:osrank} we propose a two-phase \osrank{} algorithm for
ranking projects, where the first phase uses a seed set $\seedset$ of vertices,
aimed at addressing the issue around Sybil attacks.  However, the question of
\emph{how} this set is chosen and updated remains is an open research question.

\item As discussed in Section \ref{s:oscoin}, we are committed to exploring
alternatives to proof-of-work for securing the chain.

\item A large part of the work on \osrank{} is experimentation and tuning of the
various parameters to the algorithm, \eg edge weights, damping factor, thresholds
etc. that are ongoing.

\item As it stands, \osrank{} is a great indicator of relative value
for software libraries, but doesn't provide as meaningful a metric for
user-facing applications and software used exclusively in proprietary settings.
We are looking at ways in which pairing \osrank{} with a secondary complementary
ranking system could serve a greater range of free software projects.

\end{itemize}
