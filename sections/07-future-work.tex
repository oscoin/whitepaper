\section{Future Work}
\label{s:future-work}

In our attempt to formulate a clear and concise vision for the \oscoin{}
network, we deliberately left certain questions unanswered, which we will
briefly restate and attempt to address here. The idea of \oscoin{} presented in
this work should be seen as a ``seed'' which will grow with the community and
shape itself to be the infrastructure that the free software community needs.

\subsection{Reliability of information on the ledger}
As posed in Section \ref{s:osrank}, the question of information
reliability is crucial.  We are actively researching oracle-based solutions to
this problem which could provide economic incentives for users in the network
to flag suspicious projects, as well as further disincentive projects to cheat,
by allowing the protocol to slash their unvested (\S\ref{s:vesting}) reward
balance when dishonest behavior is detected.

\subsection{Evolving and growing the seed set}
In Section \ref{s:osrank} we propose a two-phase \osrank{} algorithm for
ranking projects, where the first phase uses a seed set $\seedset$ of vertices,
aimed at addressing the issue around Sybil attacks.  However, the question of
\emph{how} this set is chosen and updated remains. A solution we are exploring
is for these decisions to be made on-chain by an elected committee voted by
the community.

\subsection{Towards a more environmentally-friendly protocol}
As discussed in Section \ref{s:oscoin}, we are committed to exploring
alternatives to proof-of-work for securing the chain.

\subsection{Evolving \osrank{}}
A large part of the work on \osrank{} is experimentation and tuning of the
various parameters to the algorithm, \eg edge weights, damping factor, thresholds
etc. We are working on building a better intuition of how exactly these parameters
affect reward distribution, and through what mechanism and in what circumstance
can the community coordinate in tuning them.
