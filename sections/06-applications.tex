\section{Applications}

\Oscoin{} as a platform has the potential to enable novel applications for
software communities. In comparison to a general purpose computer like
Ethereum, \oscoin{} provides primitives designed for code collaboration
(Section \ref{s:ledger}) and enables developers to combine them in flexible ways.

While it is difficult to envision all the potential ways \oscoin{} might be used, we
believe the following categories of applications can benefit from \oscoin{}
as a platform.

\subsection{Governance and collective decision making}

We believe developers could leverage \oscoin{}’s ledger for collective decision
making through the use of smart contracts. These contracts can take advantage of the
network graph in order to align interests between all project participants and
incentivize further participation. In addition, smart contracts can be used
within organizations to coordinate around other contentious decisions, using a
diverse set of decision making tools.

\subsection{Trust minimization for software development processes}
With contribution histories on chain, new software development processes that
minimize trust between network participants can emerge.

Rather than relying on the weak trust model of existing centralized hosting
providers, \oscoin{} can be used to create powerful continuous integration
pipelines that use cryptography to ensure that every commit can be trusted.

This category of applications is particularly relevant to high-assurance
software such aerospace applications, biomedical firmware, or cryptocurrencies,
including \oscoin{}.

\subsection{Incentivization}
The third category of applications we foresee is the one related with
developer incentives. This might include applications that expose paid work
within the \oscoin{} network, describe and enforce service level agreements
between projects and their dependencies, or completely re-imagine crowd-funding
applications such as Patreon, taking advantage of digital money and collectibles.

\subsection{DAOs}
Finally, combining all of the above with the \oscoin{} treasury
(\S\ref{s:treasury}), the open source community will have the ability to create
decentralized autonomous organizations (DAOs) that receive continuous funding
in \oscoin{}.  These funds could be allocated to both internal and external
network participants for their contributions to the project, while facilitating
direct engagement in decision making processes that are entirely transparent,
in the true spirit of the free and open source software movement.
