% TODO: add code examples.

\section{Applications and future work}

\oscoin{} as a platform has the potential to enable novel applications for
software communities. In comparison to a general purpose computer like
Ethereum, \oscoin{} provides primitives focused towards code collaboration
(Section \ref{s:ledger}) and enables developers to combine them in flexible ways.

While it’s hard to describe all the possible ways \oscoin{} might be used, we
believe the following categories of applications can benefit from \oscoin{}’s
architecture.

\subsection{Governance and collective decision making}

In the first category, we believe that developers could leverage \oscoin{}’s
ledger in order to make collective decisions through smart contracts. Developers
will have the ability to design smart contracts that leverage the network graph in
order to align interests between all project participants and incentivize further
participation. In addition, smart contracts can be used within these organizations
in order to coordinate around other contentious decisions, using a diverse set of
decision making modules.

\subsection{Trust minimization for software development processes}
The second category of applications leverages the availability of contribution
histories on chain, in order to design new software development processes that
minimize trust between network participants.

Rather than relying on the weak trust model of existing centralized hosting
providers, \oscoin{} can be used to create powerful continuous integration
pipelines that use cryptography to ensure that every commit can be trusted.

This category of applications is particularly relevant to high-assurance
software such as cryptocurrencies and navigation systems.

\subsection{Incentivization}
The third category of applications we foresee is the one related with
developer incentives. This might include applications that expose paid work
within the \oscoin{} network, describe and enforce service level agreements
between projects and their dependencies, or completely re-imagine crowd-funding
applications such as Patreon, taking advantage of digital money and collectibles.

\subsection{DAOs}
Finally, combining all of the above with the \oscoin{} treasury
(\ref{s:treasury}), users will have the ability to create decentralized
autonomous organizations (DAOs) that receive continuous funding in \oscoin{}.
These funds could be allocated to both internal and external network
participants for their contributions to the project, while also engaging them
in the decision making process of the DAO, operating fully in the open, in the
true spirit of the free and open source software movement.

% TODO: Conclusion / future work
