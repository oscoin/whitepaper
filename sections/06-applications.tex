% TODO: add code examples.

\section{Applications and future work}

\oscoin{} as a platform has the potential to enable novel applications for
software communities. In comparison to a general purpose computer like
Ethereum, \oscoin{} provides primitives designed for code collaboration
(Section \ref{s:ledger}) and enables developers to combine them in flexible ways.

While it is difficult to envision the potential ways \oscoin{} might be used, we
believe the following categories of applications can benefit from \oscoin{}’s
architecture.

\subsection{Governance and collective decision making}

We believe developers could leverage \oscoin{}’s
ledger in order to make collective decisions through smart contracts. These contracts
can leverage the network graph in
order to align interests between all project participants and incentivize further
participation. In addition, smart contracts can be used within organizations
in order to coordinate around other contentious decisions, using a diverse set of
decision making tools.

\subsection{Trust minimization for software development processes}
With on-chain contribution histories, developers might find new software development processes that
minimize the trust required to initiate engagements between network participants.

Rather than relying on the weak trust model of existing centralized hosting
providers, \oscoin{} can be used to create powerful continuous integration
pipelines that use cryptography to ensure that every commit can be trusted.

This category of applications is particularly relevant to high-assurance
software such aerospace applications, biomedical firmware, or cryptocurrencies.

\subsection{Incentivization}
We also forsee developer incentives as an area rich for experimentation.
This might include applications that (a) expose paid work
within the \oscoin{} network, (b) describe and enforce service level agreements
between projects and their dependencies, or (d) completely re-imagine crowd-funding
applications such as Patreon, taking advantage of digital money and collectibles.

\subsection{DAOs}
Finally, combining all of the above with the \oscoin{} treasury
(\ref{s:treasury}), the open source community will have the ability to create decentralized
autonomous organizations (DAOs) that receive continuous funding in \oscoin{}.
These funds could be allocated to both internal and external network
participants for their contributions to the project, while facilitating direct engagement
in DAO decision making process that are entirely transparently, in the
true spirit of the free and open source software movement.

% TODO: Conclusion / future work
