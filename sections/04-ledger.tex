\section{The Oscoin Ledger}

\subsection{Projects}
\label{s:projects}

A project $P$ is described by the tuple:
\[
    \tuple{P_{a}, P_h, P_s}
\]
where $P_{a}$ is the project's address and unique identifier, $P_h$ is
the project's current hash and $P_s$ is the canonical project source URL.

The project address is used to identify the project, as well to send
\oscoin{} to it. Each project has an account balance identified by project
address. The project hash is a digest of the project's source code at the time
it is entered in the ledger. As we'll see later, it is used to verify proofs
on the project's source code, as well as to coordinate around projects. Finally,
the project URL is there for convenience, as a means to fetch the source code.
It must be noted that the source code retrieved from $P_s$ must always hash to
$P_h$, otherwise the project is considered invalid.

\subsubsection{Registration} Projects need to be registered on the ledger
before they can participate in the network. This can be done with the
\[
    \tx{register}{P, d}
\]
transaction, where $d$ is a small deposit in \oscoin{} to prevent abandonned
projects from accumulating.

When a project is no longer in use, it is possible to unregister it from the
ledger with
\[
    \tx{unregister}{P_a, w_a},
\]
where $w_a$ is the withdrawal address for the deposit.

Projects registered on the ledger can be retrieved from the state $\state$ by
using the project address, formally: $\state(P_a) \to P$.

\subsubsection{Checkpointing} Any project in active development will see its
source code change regularly. This means the project hash $P_h$ will quickly
become out of sync with the project's latest state and will need updating. This
is done via the
\[
    \tx{checkpoint}{P_a, h, s}
\]
transaction, where $h$ is the new project hash, and $s$ is the URL to retrieve
the source code from.

\subsection{Dependencies}
\label{s:dependencies}

A project $P^1$ depends on a project $P^2$ if it references $P^2$
or parts of $P^2$ in its source code. Formally, we represent this dependency as
$P^1 \to P^2$.

Dependencies are recorded on the ledger with the
\[
    \tx{depend}{P^1, P^2}
\]
transaction, which records a dependency between $P^1$ and $P^2$.
