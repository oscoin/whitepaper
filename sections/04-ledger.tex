\section{The Oscoin Ledger}

\subsection{Projects}
\label{s:projects}

A project $P$ is described by the tuple:
\[
    \tuple{P_{a}, P_h, P_s}
\]
where $P_{a}$ is the project's address and unique identifier, $P_h$ is
the project's current hash and $P_s$ is the canonical project source URL.

The project address is used to identify the project, as well to send
\oscoin{} to it. Each project has an account balance identified by project
address. The project hash is a digest of the project's source code at the time
it is entered in the ledger. As we'll see later, it is used to verify proofs
on the project's source code, as well as to coordinate around projects. Finally,
the project URL is there for convenience, as a means to fetch the source code.
It must be noted that the source code retrieved from $P_s$ must always hash to
$P_h$, otherwise the project is considered invalid.

\subsubsection{Registration} Projects need to be registered on the ledger
before they can participate in the network. This can be done with the
\[
    \tx{register}{P, d}
\]
transaction, where $d$ is a small deposit in \oscoin{} to prevent abandonned
projects from accumulating.

When a project is no longer in use, it is possible to unregister it from the
ledger with
\[
    \tx{unregister}{P_a, w_a},
\]
where $w_a$ is the withdrawal address for the deposit.

Projects registered on the ledger can be retrieved from the state $\state$ by
using the project address, formally: $\state(P_a) \to P$.

\subsubsection{Checkpointing} Any project in active development will see its
source code change regularly. This means the project hash $P_h$ will quickly
become out of sync with the project's latest state and will need updating. This
is done via the
\[
    \tx{checkpoint}{P_a, h, s, c}
\]
transaction, where $h$ is the new project hash, $s$ is the URL to
retrieve the source code from, and $c$ is hash-linked-list of
\emph{contributions}, which are tuples:
\[
   \tuple{\field{c}{prev}, \field{c}{commit}, \field{c}{author}, \field{c}{signoff}}
\]
where:
\begin{itemize}
\item $\field{c}{prev}$ is the hash of the previous contribution, or
  $\varnothing$ if this is the first contribution to the whole
  project. Note that $c$'s first item must be linked to the last
  contribution in the project's previous checkpoint.
\item $\field{c}{commit}$ is the hash of the corresponding commit,
\item $\field{c}{author}$ is the author of the contribution,
\item $\field{c}{sig}$ is author's signature of the commit hash. 
\item $\field{c}{signoff}$ is the \emph{signoff user}, which must be a
  current maintainer of the project.
\end{itemize}
A contribution must be signed by the signoff user. This signals the
the contribution has been reviewed/verified by this maintainer.

Because all changes to a project's source code are signalled by
checkpoints, it is possible to reconstruct a full hash-linked list of
contributions for the entire project. Along with the project's
repository, this gives a full historical breakdown of who authored
which code, and which project maintainer signed them off.

There are two main uses for this: First of all the contributions in a
checkpoint are used as edges in the network graph used to compute
\osrank{} (section \ref{s:osrank}).  Second, because contributions
reference users, combined with the subjective trust system (section
\ref{s:web-of-trust}), this gives a particular user a view into the
trustworthiness of the project as a whole, based on the contributions
it is built out of. A user may judge a project as trustworthy (for
example when deciding to add it as a dependency), if all the users
involved in all the contributions (or at least the signoffs) have a
relatively high trust score, according to his trust-network.

\subsection{Dependencies}
\label{s:dependencies}

A project $P^1$ depends on a project $P^2$ if it references $P^2$
or parts of $P^2$ in its source code. Formally, we represent this dependency as
$P^1 \to P^2$.

Dependencies are recorded on the ledger with the
\[
    \tx{depend}{P^1, P^2}
\]
transaction, which records a dependency between $P^1$ and $P^2$.
