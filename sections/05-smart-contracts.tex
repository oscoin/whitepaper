\section{Smart contracts}
\label{s:smart-contracts}

\newcommand{\handler}[1]{\textsc{\small#1}}

In order to ensure that funds distributed to projects by the \oscoin{}
treasury, as well as funds received as donations are spent responsibly and
transparently, the protocol sends them to a special type of account, the
\emph{project fund}, which is an account controlled by an updatable \emph{smart
contract}.
In addition, the protocol requires that transfers from the project fund only be
issued from within the fund's smart contract. When a contributor decides to
contribute code to a project, she may inspect the contract currently in place,
and under what conditions it may be updated, so that she may be assured to be
remunerated for her work.

\subsection{Definition}
A smart contract is a set of functions, or \emph{handlers}, invoked as part
of regular transaction processing. Certain transactions, such as $\mathsf{transfer}$
make use of smart contracts to extend their behavior and allow projects to
program interactions around their governance, ownership or finances.

\subsection{Receiving \oscoin{}}

When a transfer of \oscoin{} is received by a project, the smart contract
attached to that project's fund is invoked. The specific handler that is
called in the smart contract depends on the \emph{sender}: for \oscoin{} received
from the treasury system, the \handler{receiveReward} handler is called, while
for transfers received from any other source, the \handler{receiveTransfer}
handler is called. This allows projects to handle funds received by the treasury
differently than funds received as donations, for example.

\subsubsection{Rewards from the treasury}

The \handler{receiveReward} handler is invoked every $\epoch$ blocks for
projects getting a reward. The function is called with three arguments: $p$ is
a dictionary containing the project data, $r \in \posnat$ represents the amount
of \oscoin{} being awarded to the project this epoch and $k \in \mathbb{N}$ is the
current epoch. The function must return a \emph{distribution}, a set of tuples
assigning an amount of \oscoin{} to a set of accounts. The function is valid
as long as it doesn't distribute more than $r$ \oscoin{} in total. Any \oscoin{}
that isn't distributed is effectively \emph{burnt}.

When a project is first registered, \handler{receiveReward} returns the empty
distribution, that is, the entirety of the reward $r$ is burnt:
\begin{algorithmic}[0]
    \Procedure{receiveReward}{$p, r, k$}
        \State \textbf{return} $\varnothing$
    \EndProcedure
\end{algorithmic}

\noindent Once the project owners have decided on a policy, they may issue a
transaction to update this handler. For example, they may decide that all
contributors should get an equal share of the reward, with the remainder
getting deposited into the project fund:

\begin{algorithmic}[0]
    \Procedure{receiveReward}{$p, r, k$}
        \State $n \gets |\prop{p}{contributors}|$
        \State $q \gets r \sslash n$
        \State $m \gets r \mod n$
        \State $x \gets \{ (\prop{c}{addr}, q) \mid c \gets \prop{p}{contributors} \}$
        \State \textbf{return} $x \cup \{ (\prop{p}{fund}, m) \}$
    \EndProcedure
\end{algorithmic}
This makes it very easy for a potential contributor to see whether or not they
would be rewarded for their contributions to a project, and in what proportion.

\subsubsection{Transfers and donations}

When a transfer of \oscoin{} is received from a source other than the treasury,
the \handler{receiveTransfer} handler is invoked. This function takes the same
three arguments as \handler{receiveReward}, plus a fourth argument $s$ which
represents the sender or ``source'' account of the transfer. The function is
expected to return a distribution, just like \handler{receiveReward}. Initially,
this handler is set to simply transfer the funds to the project fund:
\medskip
\begin{algorithmic}[0]
    \Procedure{receiveTransfer}{$p, r, k, s$}
        \State \textbf{return} $\{ (\prop{p}{fund}, r) \}$
    \EndProcedure
\end{algorithmic}
Just like with the \handler{receiveReward} examples, it's possible to include
arbitrary logic in this handler. For example, since donations to the project
would invoke this handler, a project might want to distribute a percentrage
of each donation to is contributors, as well as its consituent members.

\subsection{Contract handler updates}

Updates to the project smart contracts are issued using the
\[
    \tx{updatecontract}{P_a, h, c, \nu, v}
\]
transaction, which must be signed by a project owner, where $h$ is the
\emph{handler} to be updated, $c$ is the code for the new handler, $v$ is a set
of \emph{votes}: a collection of signatures the owner has collected for the
update, and $\nu$ is a \emph{nonce} to prevent replay attacks with votes.

Since updating a contract may affect existing contributors, by for example
changing the share of rewards they receive for their work; the rules according
to which contract handlers can be updated are stored in the project contract
aswell, in the form of the \handler{validContractUpdate} handler.  The
conditions under which an $\mathsf{updatecontract}$ transaction is considered
valid are specified in this handler.

\handler{validContractUpdate} is a function which takes three arguments: $p$,
the project data, $h$, the name of the handler being updated, and $v$, the list
of public keys which signed votes for the contract update transaction. The
function must return a boolean value, stating whether the update is valid ($\top$)
or not ($\bot$). The default code for this handler is:
\columnbreak
\begin{algorithmic}[0]
    \Procedure{validContractUpdate}{$p, h, v$}
        \State \textbf{return} $\{ \prop{o}{addr} \mid o \gets \prop{p}{owners} \} \subseteq v$
    \EndProcedure
\end{algorithmic}
which specifies that all the current owners must sign any contract update.

In order to specify that contract updates must be accepted by a majority of all
\emph{contributors}, this handler could be updated to:
\medskip
\begin{algorithmic}[0]
    \Procedure{validContractUpdate}{$p, h, v$}
        \State $h \gets |\prop{p}{contributors}| \sslash 2$
        \State $v' \gets \{ \prop{c}{addr} \mid c \gets \prop{p}{contributors} \} \cap v$
        \State \textbf{return} $|v'| > 1 + h$
    \EndProcedure
\end{algorithmic}
If the handler returns $\top$, the contract update transaction is considered
valid, and the handler $h$ specified in the transaction is updated with the new code
$c$.

\subsection{Ad-hoc spending}

While most distribution of \oscoin{} can be automated by the \handler{receiveReward}
and \handler{receiveTransfer} handlers, it's sometimes necessary to manually
transfer \oscoin{} from the project fund to another account. For example, in
the situation where a large donation is received by a project, the project
maintainers might want to do some planning before the funds are spent.

Hence, it's possible to spend project funds with the $\mathsf{transfer}$
transaction and an additional set of parameters:
\[
    \tx{transfer}{P_a, a, n, \nu, v} \mid n \in \posnat,
\]
where $P_a$ is the project fund account to spend from, $a$ is the account to
transfer the funds to, $n$ is the amount of \oscoin{}, and the remaining two
parameters are a set of votes.

The transaction's validity is determined by the \handler{sendTransfer} handler,
which works in a similar way to \handler{validContractUpdate}. For
example, the function might specify that the project owners may spend a small
amount $x$ of \oscoin{} per month without seeking community agreement, but that
large sums need over half of the contributors and donors to agree:
\medskip
\begin{algorithmic}[0]
    \Procedure{sendTransfer}{$p, n, a, v$}
        \If{$\fn{spentThisEpoch}{p} + n > x$}
            \State $h \gets |\prop{p}{contributors} \cup \prop{p}{donors}| \sslash 2$
            \State $c' \gets \{ \prop{c}{addr} \mid c \gets \prop{p}{contributors} \}$
            \State $d' \gets \{ \prop{d}{addr} \mid d \gets \prop{p}{donors} \}$
            \State $v' \gets v \cap c' \cap d'$
            \State \textbf{return} $|v'| > 1 + h$
            \Else
            \State \textbf{return} $\top$
        \EndIf
    \EndProcedure
\end{algorithmic}

\subsection{Closing thoughts}

We have seen how smart contracts can be used to:

\begin{enumerate}
    \item Provide transparency and trust within projects, by making reward
        distribution explicit and automatic.
    \item Allow potential contributors to easily assess a project before contributing.
    \item Offer a certain stability to existing members and collaborators, by
        making changes to contracts difficult without community buy-in.
\end{enumerate}

The exact specification of the contract language, and which built-in handlers
are available, will be specified in a subsequent paper.
