\section{Applications and future work}

The oscoin platform is a domain specific blockchain / platform that has the
potential to enable novel applications for open source software communities. In
comparison to a general purpose computer like Ethereum, Oscoin provides
primitives focused towards code collaboration (see section 4) and enables
developers to combine them in flexible ways.

While it’s hard to describe all the possible ways that oscoin might be used, we
believe that the following categories of applications benefit from oscoin’s
ledger architecture. Specifically:

\begin{enumerate}
    \item Governance and collective decision making apps
    \item Trust minimization apps for software development processes
    \item Incentivization apps
\end{enumerate}

In the first category, we believe that developers could leverage oscoin’s
ability to enumerate maintainers, contributors, dependents and supporters of an
open source project in order to create decentralized autonomous organizations
(DAOs) around open source software projects. Combining the above with the
described financial support of SrcRank (see section 3), developers will have
the ability to design smart contracts that aim to align interests between all
network participants of an open source project and incentivize further
participation. In addition, smart contracts can be used within these
organizations in order to coordinate around other contentious decisions, using
a diverse set of decision making modules.

The second category of apps, leverages the availability of source code
histories on chain, in order to imagine new software development processes that
minimize trust between network participants. Rather than relying on the weak
trust model of existing centralized hosting providers, oscoin can be used to
create powerful continuous integration pipelines that use cryptography in order
to ensure that every merge commit can be trusted.

Finally, the last category of apps we forsee is the one related with
developer incentives. The experiences we imagine, include applications that a)
expose paid work within the oscoin network, b) allow users to specify bounties
with a flexible set of agreed reviewers, c) describe and enforce service level
agreements between projects and their dependents, d) issue collectibles and non
fungible tokens unique to each software development community and e) help
creators crowdfund their efforts through a unique set of interactions between
their patrons and them.
