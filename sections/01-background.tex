\section{Background}

\begin{epigraph}{The Open-Source Everything Manifesto}
    \noindent It is in this light that we must recognize that only a restoration of
    open-source culture, and all that enables across the full spectrum of
    open-source possibilities, can allow humanity to harness the distributed
    intelligence of the collective and create the equivalent of heaven on Earth
    --- in other words, a world that works for all.
\end{epigraph}

\subsection{Cryptocurrencies and money}

With the advent of digital scarcity, it became possible to economically
incentivize and remunerate network participants simply and transparently,
without need of a trusted third-party. The introduction of
Bitcoin~\cite{bitcoin}, and later Ethereum~\cite{ethereum}, led to the
proliferation of alternative cryptocurrencies that primarily compete on
network utility and the monetary policy of the assets they provision.

If Bitcoin sought to reward network operators for validating transactions,
projects such as Zcash and Decred extended this concept to incentivizing
developers and other value producers in order to further stimulate their
respective economies. Despite these experiments, until today, no cryptocurrency
has succeeded in creating sustainable incentives for developers who either
contribute directly to the protocol, or indirectly to the underlying open
source infrastructure.

This is one of the many cases of free software projects facing incentivization
challenges, a problem that has been extensively discussed in previous
literature~\cite{roads and bridges}.

\subsection{Developer incentives and sustainability}
\label{s:incentives}

To further motivate this work, let us familiarize ourselves with the
conditions in which free and open source software\footnote{In this work, we
shall use the terms \emph{open source software} and \emph{free software} interchangeably,
to mean \emph{software distributed under terms that allow users to run it
for any purpose, as well as change and re-distribute its source code.}}
is developed. Most of the software we use on a daily basis relies on free,
publicly available code. In an increasingly digitized society, free software
projects have become the foundation of digital infrastructure underpinning many
of our societal goods and services.

With the emergence of software hosting sites like GitHub and community sites
like Stack Overflow, open source became the preferred software engineering
paradigm, resulting in numerous high quality projects published in the open,
available for anyone to use. This phenomenon helped
companies reduce lead times and bring products to the market faster, as well as
recruit talent from a new pool of technologists educated on the basis of
free and open source software.

Today, many free software projects start with individuals or small groups
solving a personally, socially, or technologically relevant problem.
Reviewing their motivation for participating in the free software ecosystem,
certain themes emerge: pride in one's work, reputation, learning,
responsibility for something they believe in, and being part of a community.

While most open source software projects start for the reasons above, those
that gain momentum require significant time and financial resources to
maintain. This creates a fundamental problem: an important subset of projects
started as volunteer work are becoming critical public
infrastructure. And while some developers find ways to finance their efforts,
most struggle to keep up with community requests during their free time, often
abandoning their projects or burning out under the burden of increasing
responsibility. It is this problem, developing a healthy and sustainable means
of free software maintenance and financing, that we address in this work.
