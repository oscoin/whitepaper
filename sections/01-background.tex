\section{Background}

While cryptocurrencies introduced a solution to the problem of digital scarcity
and enumeration, they have failed to create sustainable incentives for the
developers that participate in them.

In this paper we introduce a cryptocurrency that rewards and provisions
open-source software, as well as a platform for establishing trust and
transparency in open source communities.

\subsection{Cryptocurrencies and money}

The invention of digital scarcity made it possible to economically incentivize
and remunerate network participants for their service in a simple and transparent
way, without mediation by a trusted third-party. The introduction of Bitcoin,
and later Ethereum, led to an explosion of alternative cryptocurrencies that
primarily compete on a) network utility and b) monetary policy of the economy
that they provision.

If Bitcoin sought to reward network operators for their service in validating
transactions, other projects such as Zcash and Dash extended that reward to
incentivize value producers in their networks in order to further stimulate
their economy. Despite most of these experiments, till today, no cryptocurrency
has succeeded in creating sustainable incentives for developers that want to
participate on their networks, either directly by contributing to the protocol
or indirectly by contributing to the open source infrastructure that underlies
it.

The problem is just another instance of the challenges associated with the
incentivization of free and open source software, that have been broadly
discussed and studied in literature.

\subsection{Developer incentives and the problem of open-source software
sustainability}

To further motivate this work, we must first familiarize ourselves with the
conditions in which free and open-source software is developed. Software whose
source code is publicly available is called open-source. Most of the software
we use on a daily basis relies on free and public code. In an increasingly
digitized society, these open-source projects have become the foundation of
digital infrastructure underpinning many of our societal goods and services.

With the emergence of software hosting sites like GitHub and community sites
like Stack Overflow, the open-source paradigm became the focal point of
software development, resulting in the development of numerous high quality
projects which are openly available for anyone to use. This phenomenon helped
companies (a) to reduce lead times and bring products to the market faster and
(b) to recruit talent from a new pool of technologists, educated on the basis
of open-source software.

Today, many open-source projects are started by individuals or small groups of
people solving a personally, socially, or technically relevant problem. By
analysing people's motivations for participating in open-source software
development the following themes emerge: (a) pride in one's work, (b)
reputation, (c) learning, (d) responsibility for something they believe in,(e)
being part of a community and (f) financial compensation.

While most open-source software projects start for the reasons above, the ones
that gain momentum require significant resources in time and money to maintain.
This creates a fundamental problem: projects created in the spare time of a
developer are becoming critical public infrastructure. While some developers
find ways to sustainably finance their efforts, most developers, as their
projects gain popularity, experience stress and exhaustion trying to keep up
with the requests of the community during their free time, often abandoning
their own projects or burning out under the increased responsibility. Left
unchecked, this leads to a tragedy of the commons which is at the core of the
problem this paper attempts to address: financial sustainability and
transparency in the development of open-source software

% TODO: [Add diagram with overview of existing solutions and fit with different
% types of open source work]

\subsection{The linked structure in open-source software}

While the importance of an open-source software package is highly subjective to
each individual, valuable information about its relative importance can be
extracted by studying the structure of open-source software through
dependencies.

Similarly to how Google pioneered a large-scale search engine around PageRank
that takes advantage of the present structure in hypertext (the web) in order
to organize the world’s information, the linked graph of open-source software
dependencies offers an important resource when it comes to information
extraction, that can be thought of as an implicit form of implicit trust
present in the structure.

