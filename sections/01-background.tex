\section{Background}

\begin{epigraph}{The Open-Source Everything Manifesto}
    \noindent It is in this light that we must recognize that only a restoration of
    open-source culture, and all that enables across the full spectrum of
    open-source possibilities, can allow humanity to harness the distributed
    intelligence of the collective and create the equivalent of heaven on Earth
    --- in other words, a world that works for all.
\end{epigraph}

\subsection{Cryptocurrencies and money}

With the advent of digital scarcity, it became possible to economically
incentivize and remunerate network participants simply and transparently,
without need of a trusted third-party. The introduction of
Bitcoin~\cite{bitcoin}, and later Ethereum~\cite{ethereum}, led to the
proliferation of alternative cryptocurrencies that would primarily compete on
network utility and monetary policy of the assets they provisioned.

% TODO: I am finding the following paragraph a bit odd for an opening
% paragraph. Funding the development of oscoin software isn't the fundamental
% goal. I agree that this is a cute example (it brings in all the relevant
% actors and shows that an undeniably valuable OS project isn't being funded by
% any mechanism), but maybe we should move it to nearer the end of this section,
% e.g. to give evidence to 'spare time of a developer are becoming critical
% public infrastructure'.

If Bitcoin sought to reward network operators for their services validating
transactions, projects such as Zcash and Decred extended this concept to
incentivizing developers and other value producers in order to further stimulate
their respective economies. Despite these experiments, until today, no cryptocurrency
has succeeded in creating sustainable incentives for developers who either
contribute directly to the protocol, or indirectly to the underlying open source
infrastructure.

The problem is another iteration of the challenges associated with the
incentivization of free and open source software that have been broadly
discussed and studied in literature~\cite{roads and bridges}.

\subsection{Developer incentives and sustainability}
\label{s:incentives}

To further motivate this work, let us familiarize ourselves with the
conditions in which free and open source software is developed. Most of the
software we use on a daily basis relies on free, publicly available code. In an
increasingly digitized society, open source projects have become the
foundation of digital infrastructure underpinning many of our societal goods
and services.

With the emergence of software hosting sites like GitHub and community sites
like Stack Overflow, the open source paradigm became the focal point of
software engineering, resulting in the development of numerous high quality
projects which are openly available for anyone to use. This phenomenon helped
companies reduce lead times and bring products to the market faster as well as
recruit talent from a new pool of technologists, educated on the basis of
open source software.

Today, many open source projects start with individuals or small groups
solving a personally, socially, or technologically relevant problem. By
looking at people's motivations for participating in the open source ecosystem,
certain themes emerge: pride in one's work, reputation, learning,
responsibility for something they believe in, being part of a community, and
financial compensation.

While most open source software projects start for the reasons above, those
that gain momentum require significant time and financial resources to
maintain.  This creates a fundamental problem: an important subset of projects
started in a developer's spare time are becoming critical public
infrastructure. And while some developers find ways to finance their efforts,
most struggle to keep up with community requests during their free time, often
abandoning their projects or burning out under the increased burden of
responsibility. It is this problem of developing a healthy, sustainable means
for open source maintenance and financing that we address in this work.
