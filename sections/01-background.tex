\section{Background}

\begin{epigraph}{The Open-Source Everything Manifesto}
    \noindent It is in this light that we must recognize that only a restoration of
    open-source culture, and all that enables across the full spectrum of
    open-source possibilities, can allow humanity to harness the distributed
    intelligence of the collective and create the equivalent of heaven on Earth
    --- in other words, a world that works for all.
\end{epigraph}

\subsection{Cryptocurrencies and money}

The advent of digital scarcity made it possible to economically incentivize
and remunerate network participants for their service, simply and
transparently, without need of a trusted third-party. The introduction
of Bitcoin~\cite{bitcoin}, and later Ethereum~\cite{ethereum}, led to the
proliferation of alternative cryptocurrencies that primarily compete on network
utility and monetary policy of the assets they provision.

% TODO: I am finding the following paragraph a bit odd for an opening
% paragraph. Funding the development of oscoin software isn't the fundamental
% goal. I agree that this is a cute example (it brings in all the relevant
% actors and shows that an undeniably valuable OS project isn't being funded by
% any mechanism), but maybe we should move it to nearer the end of this section,
% e.g. to give evidence to 'spare time of a developer are becoming critical
% public infrastructure'.

If Bitcoin sought to reward network operators for their services validating
transactions, projects such as Zcash and Decred extended this concept to
incentivizing developers and other value producers in order to further stimulate
their respective economies. Despite these experiments, until today, no cryptocurrency
has succeeded in creating a sustainable funding structure for developers who either
contribute directly to the protocol, or indirectly to the underlying open source
infrastructure.

% TODO: The point in the last sentence is perhaps arguable. Why doesn't Zcash do this
% for instance?

The is one example among many within the broader free and open source software
sector, where sustainable incentive alignment has become a central subject of discussion
and research~\cite{roads and bridges}.

\subsection{Developer incentives and sustainability}
\label{s:incentives}

To further motivate this work, let us familiarize ourselves with the
conditions in which free and open source projects are developed. Most of the software
we use on a daily basis relies on free, publicly available code. In an increasingly
digitized society, open source projects are the foundation of
a shared digital infrastructure that now supports many vital societal goods and services.

Open source publishing introduced a model in which a self-organized network of peers
could collaborate at global scale. Facilitated by software hosting platforms like GitHub
and community forums like Stack Overflow, the open source paradigm came into prominence
for the simple reason that sharing code was easier and more efficient
in a world of open collaboration and peer-to-peer distribution.

With numerous high quality projects free for anyone to use,
companies have reduced lead times and brought products to market
faster, while recruiting from a new pool of technologists educated on the basis of
open source development.

Today, many open source projects start with individuals or small groups
solving a personally, socially, or technologically relevant problem. By
analysing people's motivations for participating in open source software
development the following themes emerge: (a) pride in one's work, (b)
reputation, (c) learning, (d) responsibility for something they believe in, (e)
being part of a community and (f) financial compensation.

% TODO Listing people's motivations is a little impersonal. Consider turning this into a sentence.

While most open source software projects start for the reasons above, those
that gain momentum require significant time and financial resources to maintain.
This creates a fundamental problem: projects that start in developers' spare time
are becoming critical public infrastructure. And while some developers
find ways to sustainably finance their efforts, most struggle to keep up with
community requests during their free time, often abandoning
projects or burning out under the increasing pressure of continued upkeep.
It is this problem of developing a healthy, sustainable means for open source
maintinence that this manuscript seeks to address.

% TODO: [Add diagram with overview of existing solutions and fit with different
% types of open source work]
