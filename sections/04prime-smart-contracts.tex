\section{Smart contracts}

The \oscoin{} blockchain uses smart contracts as a way to distribute funds that
a project has received. These smart contracts are added with special
transactions which modify the behaviour of a project. For example to control how
funds from the core mechanism payouts are distributed, a project could use the
following code:

\begin{lstlisting}
(fn [proj reward]
  (let [cs    (lookup :contributors p)
        n     (length cs)
        share (div reward n)]
    (map (fn [c]
           [:transfer (lookup :address p) (lookup :address c) share])
         cs)))
\end{lstlisting}

This is a function which is invoked every time project $P$ gets a network
reward. The argument \texttt{proj} is a dictionary containing the project
metadata, while \textt{reward} is a natural number representing the amount of
oscoin that was awarded to the project. The function must return a list
transactions.
