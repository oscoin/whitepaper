% TODO(james):
% - merge with ledger sections
% - add bounty example

\section{Smart contracts}
\label{s:smart-contracts}

\newcommand{\handler}[1]{\textsc{\small#1}}

In order to ensure that funds distributed to projects by the \oscoin{}
treasury, as well as funds received as donations are spent responsibly and
transparently, the protocol sends them to a special type of account, a
\emph{project fund}, which is an account controlled by an updatable \emph{smart
contract}.
In addition, the protocol requires that transfers from the project fund only be
issued from within the fund's smart contract. When a contributor decides to
contribute code to a project, she may inspect the contract currently in place,
and under what conditions it may be updated, so that she may be assured to be
remunerated for her work.

\subsection{Distributing treasury rewards}

The most important function in the project fund smart contract,
\handler{receiveTransfer} is in charge of handling \oscoin{} received from the
treasury system. This function is invoked every epoch for projects getting a
reward. The function is called with two arguments: $p$ is a dictionary
containing the project data and $r \in \posnat$ represents the amount of
\oscoin{} that was awarded to the project this epoch. The function must return
a \emph{distribution}, a set of tuples assigning a number of \oscoin{} to
various accounts. The protocol checks the function doesn't distribute more than
$r$.

When a project is first registered, \handler{receiveTransfer} returns the empty
distribution, that is, the \oscoin{} is burnt:
\begin{algorithmic}[0]
    \Procedure{receiveTransfer}{$p, r$}
        \State \textbf{return} $\varnothing$
    \EndProcedure
\end{algorithmic}

% TODO: receiveTransfer should take "sender" to distinguish donations from
% rewards.
Once the owners have decided on a policy, they may issue a transaction to update
this function. For example, the project owners may decide that all contributors
should get an equal share of the reward, with the remainder getting deposited
into the project fund:
\medskip
\begin{algorithmic}[0]
    \Procedure{receiveTransfer}{$p, r$}
        \State $n \gets |\prop{p}{contributors}|$
        \State $s \gets r \sslash n$
        \State $m \gets r \mod n$
        \State $x \gets \{ (\prop{c}{addr}, s) \mid c \gets \prop{p}{contributors} \}$
        \State \textbf{return} $x \cup \{ (\prop{p}{fund}, m) \}$
    \EndProcedure
\end{algorithmic}

\subsection{Contract function updates}

Contract updates are issued using
\[
    \tx{updatecontract}{P_a, h, c, \nu, v}
\]
transactions, signed by a project owner, where $h$ is the \emph{handler} to be
updated, $c$ is the code for the new function, $v$ is a set of \emph{votes}: a
collection of signatures the owner has collected for the update, and $\nu$ is a
\emph{nonce} to prevent replay attacks with votes. The previous function is an
example that would be used for the \handler{receiveTransfer} handler. The
transaction is only valid if the \handler{validContractUpdate} handler
described next returns $\top$.

Of course the owners cannot just update contracts whenever they want, to
whatever they like. Under which circumstances a contract can be updated is
controlled by another handler: \handler{validContractUpdate}. This is a
function which takes two arguments: $p$, the project data, and $v$, the list of
public keys which signed votes for the contract update transaction. The
function must return a boolean value, stating if the update is valid or not.
The default code of this handler is:
\begin{algorithmic}[0]
    \Procedure{validContractUpdate}{$p, v$}
        \State \textbf{return} $\{ \prop{o}{addr} \mid o \gets \prop{p}{owners} \} \subseteq v$
    \EndProcedure
\end{algorithmic}
which specifies that all the current owners must sign any contract update.

In order to specify that contract updates must be accepted by a majority of all
\emph{contributors}, this handler could be updated to:
\medskip
\begin{algorithmic}[0]
    \Procedure{validContractUpdate}{$p, v$}
        \State $h \gets |\prop{p}{contributors}| \sslash 2$
        \State $v' \gets \{ \prop{c}{addr} \mid c \gets \prop{p}{contributors} \} \cap v$
        \State \textbf{return} $|v'| > 1 + h$
    \EndProcedure
\end{algorithmic}

% TODO: Section on donations still needed?

\subsection{Ad-hoc spending}

The handler \handler{receiveTransfer} is useful for declaring a project's usual policy for distributing \oscoin{}
(\eg{} paying for bug-fixes and small contributions, etc.) However it is likely
that large donations will be redirected to the project fund rather than be
immediately distributed, so that more planning can take place before the funds
are spent.

To spend \oscoin{} from the project fund, $\mathsf{transfer}$ can be used with an
additional set of parameters:
\[
    \tx{transfer}{P_a, n, \nu, v} \mid n \in \posnat
\]
where $v$ is a list of votes and $\nu$ is a nonce.

The transaction is valid if the handler \handler{sendTransfer} accepts
it. This function works in a similar way to \handler{validContractUpdate}. For
example, the function might specify that the project owners may spend a small
amount of \oscoin{} per month without seeking community agreement, but that
large sums need over half of contributors and donors to agree:
\medskip
\begin{algorithmic}[0]
    \Procedure{sendTransfer}{$p, n, v$}
        \If{$\fn{spentThisEpoch}{p} + n > 100$}
            \State $h \gets |\prop{p}{contributors} \cup \prop{p}{donors}| \sslash 2$
            \State $c' \gets \{ \prop{c}{addr} \mid c \gets \prop{p}{contributors} \}$
            \State $d' \gets \{ \prop{d}{addr} \mid d \gets \prop{p}{donors} \}$
            \State $v' \gets v \cap c' \cap d'$
            \State \textbf{return} $|v'| > 1 + h$
            \Else
            \State \textbf{return} $\top$
        \EndIf
    \EndProcedure
\end{algorithmic}

The exact specification of the contract language, and which built-in functions
are available, will be specified in a subsequent paper.
