\section{Project smart contracts}

Most transactions relating to a project can be signed by any one of the
project's owners. This includes transferring funds to and from the project's
account. In order to ensure that funds distributed through \osrank{} (and
possibly donations) is spent in a responsible manner, the protocol requires that
transfers from project accounts be accepted by a \emph{smart contract}, a
function written in a special purpose programming language which triggers when
certain events arise. When a contributor decides to contribute code to a
project, she may inspect the contracts currently in place (and under what
conditions they may be updated, etc.), so that she may be assured to be
renumerated for her work.
 
For the language, we will use a modification of \radicle{}, the language we have
developped for P2P code collaboration. The main characteristics are:
\begin{itemize}
\item Lisp-like, with Clojure-like syntax,
\item Completely deterministic,
\item Reduction-counting, for preventing unbounded computations.
\end{itemize}

The most important contract for a project is the one used to distribute rewards
to contributors. This function is invoked every epoch, if the project gets a
payout according to it's \osrank{}. The function is called with two arguments:
\texttt{proj} is a dictionary containing the project data and \texttt{reward} is
a natural number representing the amount of oscoin that was awarded to the
project that epoch. The function must return a list transactions which the
project is authorised to make.

When a project is first registered it's reward distribution function is set to
the default:

\begin{algorithmic}[1]
\Procedure{onReward}{$p, r$}
\EndProcedure
\end{algorithmic}

This means that no transactions will take place; funds will just accumalte in
the project fund.

Once the owners have decided on a policy, they may issue a transaction to update
this function. For example, the project owners may decide that all contributors
should get an equal share of the reward:

\begin{algorithmic}[1]
\Procedure{onReward}{$p, r$}
\State $s \gets r \sslash \fn{count}{\prop{p}{contributors}}$
\For{$c \gets \prop{p}{contributors}$}
  \State $\tx{transfer}{\prop{p}{addr}, \prop{c}{addr}, s}$
\EndFor
\EndProcedure
\end{algorithmic}

Code updates are issued using
\[
    \tx{projectcontract}{s, c, v}
\]
transactions, signed by a project owner, where $s$ is the \emph{slot} to be
updated, $c$ is the code for the new function, and $\mathtt{v}$ is a set of
\emph{votes}: a collection of signatures the owner has collected for the
update. The previous function is an example that would be used for the
$\mathtt{onReward}$ slot.

Of course the owners cannot just update contracts whenever they want, to
whatever they like. How contracts can be updated is controlled by another slot:
$\mathtt{validContractUpdate}$. This is a function which takes 3 arguments:
$\mathtt{proj}$, the project data, $\mathtt{votes}$, the list of public keys
which signed votes for the contract update transaction. The default code at this
slot is:

\begin{algorithmic}[1]
\Procedure{validContractUpdate}{$p, v$}
\State \textbf{return} $\{ \prop{o}{addr} \mid o \gets \prop{p}{owners} \} \subseteq v$
\EndProcedure
\end{algorithmic}
which specifies that all the current owners must sign any contract update.

In order to specify that contract updates must be accepted by a majority of all
contributors, the code at this slot could be updated to:

\begin{algorithmic}[1]
\Procedure{validContractUpdate}{$p, v$}
\State $h \gets \fn{count}{\prop{p}{contributors}} \sslash 2$
\State $v' \gets \{ \prop{c}{addr} \mid c \gets \prop{p}{contributors} \} \cap v$
\State \textbf{return} $\fn{count}{v'} > 1 + h$
\EndProcedure
\end{algorithmic}

The exact specification of the contract language, and which built-in functions
are available, will be specified in a subsequent paper.
